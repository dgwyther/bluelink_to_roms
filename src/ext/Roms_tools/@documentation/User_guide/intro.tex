This guide presents a series of Matlab routines that 
could be useful for the pre- and post-processing of oceanic
regional ROMS simulations.

The Regional Ocean Modeling System (ROMS) is a new
generation ocean circulation model \citep{Shc03b}
that has been specially designed for accurate simulations
of regional oceanic systems.
The reader is referred to \citet{Shc03a} and to 
\citet{Shc03b} for a complete description of the model.
ROMS has been applied for the regional simulation of a 
variety of different regions of the world oceans 
\citep[e.g.][]{Bla02,Dil03,Hai00,Mac02,Mar03,Pen01}.

To perform a regional simulation using ROMS, the modeler needs
to provide a whole set of data in a specific format:
horizontal grid, topography, surface forcing,
lateral boundary conditions. He also needs to analyze
the model outputs. The tools that are described here
have been designed to perform these tasks.
The goal is to be able to build a standard
regional model configuration in a minimum time. 

In the first chapter, the system requirements and
the installation process are exposed. 
A short  note on ROMS model is presented in 
chapter 2.
A tutorial on the use of
ROMSTOOLS is shown in the third chapter.
Tidal and nesting tools are presented in chapter 4 and 5.

\subsection{Installation}

\subsubsection{System requirement}

This toolbox has been designed for Matlab.
It needs at least 2 Gbites of disk space.
It has been tested on Linux workstations, but it 
can be used on any platform if a Matlab netcdf mex file is
provided.
A Matlab netcdf mex file can be found at 
the location: \\
{\small
http://woodshole.er.usgs.gov/staffpages/cdenham/public\_html/MexCDF/nc4ml5.html\#DOWNLOADING
}. All the necessary Matlab toolboxes (netcdf and m\_map) are included 
in the ROMSTOOLS package. The topography \citep{Smi97}, 
hydrography \citep{Con02}, 
and surface fluxes \citep{Das94}
global datasets are also included.

\subsubsection{Getting the files}

A compressed tar file (roms\_tools.tar.gz) containing the 
Matlab programs and all the necessary
datasets (and also a version of ROMS) is located at :
\begin{center}
http://fraise.univ-brest.fr/$\sim$penven/Roms\_tools/ 
\end{center}
Be careful, the file is large ($\sim$ 530 Megs).
If you are unable to obtain the file in this fashion, feel 
free to contact me at : \\ 

Pierrick Penven

Institut de Recherche pour le D\'eveloppement

Laboratoire de Physique des Oc\'eans

Boite Postale 809

6, Avenue Le Gorgeu

29285 Brest CEDEX

FRANCE \\
\\
email: Pierrick.Penven@ird.fr
\\

\noindent It is also possible to get ROMSTOOLS in three separate tar files:
\begin{itemize}
\item data.tar.gz for the datasets
\item tools.tar.gz for the Matlab programs
\item Roms\_agrif.tar.gz for the ROMS source code.
\end{itemize}
This way, it is not necessary to download again the whole dataset when a new ROMS
release is available.

\subsubsection{Extracting the files}

Uncompress and untar the file (gunzip and tar -xvf). You should get the 
following directory tree : \\
\\
Roms\_tools

$|$-- COADS05 

$|$\hspace{0.5cm}$|$-- Matlab

$|$-- Compile

$|$-- Diagnostic\_tools

$|$-- Documentation

$|$\hspace{0.5cm}$|$-- Matlab

$|$-- m\_map

$|$\hspace{0.5cm}$|$-- private

$|$-- Nesting\_tools

$|$-- netcdf\_g77

$|$-- netcdf\_ifc

$|$-- netcdf\_matlab

$|$\hspace{0.5cm}$|$-- listpick

$|$\hspace{0.5cm}$|$-- ncatt

$|$\hspace{0.5cm}$|$-- ncbrowser

$|$\hspace{0.5cm}$|$-- ncdim

$|$\hspace{0.5cm}$|$-- ncfiles

$|$\hspace{0.5cm}$|$-- ncitem

$|$\hspace{0.5cm}$|$-- ncrec

$|$\hspace{0.5cm}$|$-- nctype

$|$\hspace{0.5cm}$|$-- ncutility

$|$\hspace{0.5cm}$|$-- ncvar

$|$\hspace{0.5cm}$|$-- netcdf

$|$-- Preprocessing\_tools

$|$-- Roms\_Agrif

$|$\hspace{0.5cm}$|$-- AGRIFZOOM

$|$\hspace{0.5cm}$|$\hspace{0.5cm}$|$-- AGRIF\_FILES

$|$\hspace{0.5cm}$|$\hspace{0.5cm}$|$-- AGRIF\_INC

$|$\hspace{0.5cm}$|$\hspace{0.5cm}$|$-- AGRIF\_OBJS

$|$\hspace{0.5cm}$|$\hspace{0.5cm}$|$-- AGRIF\_YOURFILES

$|$\hspace{0.5cm}$|$\hspace{0.5cm}$|$-- CVS

$|$\hspace{0.5cm}$|$\hspace{0.5cm}$|$-- LIB.clean

$|$\hspace{0.5cm}$|$-- CVS

$|$-- Run

$|$-- Tides

$|$-- Topo

$|$\hspace{0.5cm}$|$-- Matlab

$|$-- Visualization\_tools

$|$-- WOA2001
\\
\\
Meaning of the different directories :
\begin{itemize}
\item COADS05 : Location of the COADS surface fluxes data
(monthly climatology at $0.5^\circ$ resolution).
\item Compile : Empty scratch directory for ROMS compilation.
\item Diagnostic\_tools : A few Matlab scripts for animations and
statistical analysis.
\item Documentation
\item m\_map : The Matlab mapping toolbox 
(http://www2.ocgy.ubc.ca/$\sim$rich/map.html).
\item Nesting\_tools : preprocessing tools used to prepare nested
grids.
\item netcdf\_g77 : The netcdf Fortran library for Linux compiled with g77\\
(http://www.unidata.ucar.edu/packages/netcdf/index.html).
\item netcdf\_ifc : The netcdf Fortran library for Linux compiled with ifc. The
Intel Fortran 95 compiler (ifc) is available at \\
http://www.intel.com/software/products/compilers/flin/noncom.htm.
\item netcdf\_matlab : The Matlab netcdf toolbox  \\
({\small
http://woodshole.er.usgs.gov/staffpages/cdenham/public\_html/MexCDF/nc4ml5.html}).
This is where the mexfile (mexcdf60.mexglx) is located (this needs 
to be replaced if you are not using a Linux PC).
\item Preprocessing\_tools : preprocessing Matlab scripts.
\item Roms\_Agrif : ROMS Fortran sources.
\item Run : Working directory. This is where the ROMS input files
are generated and where the model is running.
\item Tides : Location of the tidal dataset and the Matlab routines 
to prepare ROMS tidal simulations.
The tidal dataset  is derived from the Oregon State University
global model of ocean tides TPXO.6 \citep{Egb02}: 
http://www.oce.orst.edu/po/research/tide/global.html
\item Topo : Location of the global topography dataset at $2'$ resolution
\citep{Smi97}. Original data can be found at:
http://topex.ucsd.edu/cgi-bin/get\_data.cgi
\item Visualization\_tools : Matlab scripts for the ROMS visualization
graphic user interface.
\item WOA2001 : World Ocean Atlas 2001 global dataset 
(monthly climatology at $1^\circ$ resolution).
\end{itemize}

If you need to create and play .fli animations, you should install ppm2fli
and xanim on your system. If you have a Linux PC, you can follow these steps:
\begin{enumerate}
\item log in as root
\item go to the Roms\_tools directory
\item type : rpm -Uvh  ppm2fli-2.1-1.i386.rpm
\item type : rpm -Uvh  xanim-2.80.1-12.i386.rpm
\item log out
\end{enumerate}
If you are not using a Linux PC, you should ask your 
system administrator to install these programs.


\subsection{Future plans}
\begin{itemize}
\item A graphic user interface should be useful for the
preprocessing tools.
\end{itemize}

\subsection{Warnings and bugs}
\begin{itemize}
\item Since Geostrophy is used to derived horizontal currents 
from hydrology data, this method can not be applied close to 
the equator. To perform equatorial simulations (in a 
$5^\circ$S-$5^\circ$N band), the user should find another 
way to provide horizontal currents at model boundaries.
\item The interpolation procedure does not handle grids that cross
the 180$^\circ$E longitude meridian. 
Grids should stay between longitudes -180$^\circ$E and 180$^\circ$W,  
and avoid the middle of the Pacific Ocean.
\item On extended grids, the objective analysis used for data 
extrapolation can be relatively costly in memory and CPU time. 
The "nearest" Matlab function that is less costly is used instead.
If the computer starts to swap, you should think of reducing the 
dimension of the model's domain.
\end{itemize}
