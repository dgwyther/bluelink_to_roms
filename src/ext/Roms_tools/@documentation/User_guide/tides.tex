Using the methods described by  \citet{Fla76}, ROMS is able to propagate the different 
tidal constituents from its boundaries. Simply  define  the cpp keys SSH\_TIDES
and UV\_TIDES and recompile the model using jobcomp.

\subsection{Preparing the tides boundary conditions}

The tidal components are added to the forcing file (XXX\_frc.nc)
by the Matlab program make\_tides.m.
Edit the file : $\sim$/Roms\_tools/Preprocessing\_tools/make\_tides.m.
The part of the file that you should change is :\\
\\
\%\\
\% TPXO file name\\
\%\\
tidename='../Tides/TPXO6.nc';\\
\%\\
\% ROMS file names\\
\%\\
gname='roms\_grd.nc';\\
fname='roms\_frc.nc';\\
\%\\
\% Number of tides component to process\\
\%\\
Ntides=5;\\
\%\\
\% Set start time of simulation\\
\%\\
year=2000;\\
month=1;\\
day=15;\\
hr=0.;\\
minute=0.;\\
second=0.;\\
\\
The variables that you can change :
\begin{itemize}
\item tidename='../Tides/TPXO6.nc'; : Location of the netcdf tidal dataset. 
This file is derived from 
the Oregon State University global model of ocean tides TPXO.6 \citep{Egb02}. 
Data sources can be found at 
http://www.oce.orst.edu/po/research/tide/global.html.
\item gname='roms\_grd.nc' : Name of the ROMS netcdf grid file.
\item fname='roms\_frc.nc' : Name of the ROMS netcdf forcing file. This is where 
the tidal informations will be added.
\item Ntides=5; : Number of tidal components to process. Be careful !
This value should be identical to the value of the parameter Ntides in param.h:
"parameter (Ntides=5)".
\item year=2000; month=1; day=15; hr=0.; minute=0.; second=0.; : This is the 
starting time of simulation. A procedure correct phases and amplitudes for real 
time runs. It employs parts of a post-processing code from \citet{Egb02} TPXO model.
\end{itemize}

\subsection{Run the model}

Save make\_tides.m and run it in the Matlab session. The model is now ready to run 
with tides. To work correctly, the model should use the \citet{Fla76} open boundary 
radiation scheme (cpp key OBC\_M2FLATHER defined).
