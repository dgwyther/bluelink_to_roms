%
% nesting
%

\subsection{Introduction}

To address the challenge of bridging the gap between near-shore and
offshore dynamics, a nesting capability has been added to ROMS
and tested for the California Upwelling System \citep{Pen04}.
The method chosen for embedded griding takes advantage of the AGRIF
(Adaptive Grid Refinement in Fortran) package \citep{Bla99,Deb00,
Deb03a,Deb03b}.
AGRIF is a Fortran 95 package for the inclusion of adaptive mesh refinement
(AMR) features within a finite difference numerical model.  The
package is based on the use of pointers which minimizes the changes in
the original numerical model.  Regardless of the possibility of
adaptive refinement (i.e the automatic generation and evolution of
high-resolution subdomains in response to dynamical criteria), one of
the major advantages of AGRIF in static-grid embedding is the ability to
manage an arbitrary number of fixed grids and an arbitrary number of
embedding levels.

\begin{figure}[htbp]
\centerline{\psfig{figure=emb_fig1.eps,width=16cm}}
\caption{Temporal coupling between a parent and a child grid
for a refinement factor of 3.  The coupling is done at the baroclinic
time step.}
\label{fig:temp_coupling}
\end{figure}

A recursive integration procedure manages the time evolution for the
child grids during the time step of the parent grids
(Fig. \ref{fig:temp_coupling}).  ROMS resolves explicitly the external
mode.  In order to preserve the CFL criterion, for a typical
coefficient of refinement (say, a factor of 3 for a 5 km resolution
grid embedded in a 15 km grid), for each parent time step the child
must be advanced using a time step divided by the coefficient of
refinement as many time as necessary to reach the time of the parent
(Fig.  (\ref{fig:temp_coupling})).  For simple 2-level embedding, the
procedure is as follows:
\begin{enumerate}
\item Advance the parent grid by one parent time step.
\item Interpolate the relevant parent variables in space and time
to get the boundary conditions for the child grid.
\item Advance the child grid by as much child time steps as necessary
to reach the new parent model time.
\item Update point by point the parent model by averaging the more
accurate values of the child model (in the case of 2-way embedding).
\end{enumerate}
The recursive approach allows the specification of any number
of embedding level.

\subsection{Prepare the child model}

To run an embbeded model, the user must provide the grid, the surface 
forcing and the initial conditions. To name the different files
AGRIF employs a specific strategy: if the parent file names are of
the form: XXX.nc, the first child names will be of the form: 
XXX.nc.1, the second: XXX.nc.2, etc... 
This convention is also applied to the roms.in files.

A graphic user interface (NestGUI) facilitates the generation of the different 
netcdf files. Launch nestgui in the Matlab session 
(in $\sim$/Roms\_tools/Run/ directory):
\\ \\
$>>$ nestgui
\\ \\
\begin{figure}[!ht]
\centerline{\psfig{figure=nestgui1.eps,width=7cm}}
\caption{Entrance window of NestGUI}
\label{fig:nestgui1}
\end{figure}
A window pops up, asking for a "PARENT GRID" netcdf file 
(Figure \ref{fig:nestgui1}). In our Peru test case, you should select 
roms\_grd.nc (grid file) and click "open".
The main window appears (Figure \ref{fig:nestgui2}). 
\begin{figure}[!ht]
\centerline{\psfig{figure=nestgui2.eps,width=12cm}}
\caption{The NestGUI main window}
\label{fig:nestgui2}
\end{figure}
\begin{enumerate}
\item To define the child domain, click "Define child" and click on 2 opposite corners
of the child domain in the main window. The size of the grid child
(Lchild and Mchild) are now visible. This operation can be redone 
until you are happy with the position of the child domain.
\item Click "Interp child". You will be asked if you want to use a new topography.
If the answer is "yes", you will have to give the location of the topography file:
$\sim$/Roms\_tools/Topo/etopo2.nc. The bathymetry will be smoothed until it 
fits the r factor \citep{Bec93} that is defined in the NestGUI window.
It will be connected to the parent topography following the relation:
$h_{new}=\alpha.h_{child} + (1-\alpha).h_{parent}$, where $\alpha$ is going from 0
to 1 in n points form the boundary (also defined in the NestGUI window). 
If the answer is "no", the parent topography is simply interpolated 
on the child grid.
\item Click "Interp forcing". The parent surface fluxes are interpolated on
the child grid.
\item Click "Interp initial". The parent initial conditions are interpolated on
the child grid. In case of different topographies between the parent and 
the child grids, the child initial conditions are vertically re-interpolated.
\end{enumerate}

"Interp clim" might be useful to generate boundary conditions to test the child 
model alone. "Interp restart" is used to generate a child restart file. This
a child domain can be "hot started" after the spin-up of the parent model.

\subsection{Compile and Run}

The ROMS nesting procedure needs a Fortran 95 compiler. For Linux PCs,
the Intel Fortran 95 Compiler (IFC) is available for free at \\
http://www.intel.com/software/products/compilers/flin/noncom.htm.
To be able to compile ROMS with IFC, you should change the corresponding 
comments in jobcomp. Define AGRIF in \\
$\sim$/Roms\_tools/Run/cppdefs.h.
Other cppkeys are related to AGRIF:
\begin{itemize}
\item AGRIF\_STORE\_BAROT\_CHILD : Store ubar and vbar during the parent step for the
child boundary conditions (Nesting).
\item AGRIF\_FLUX\_BC : Apply parent/child barotropic boundary conditions has 
fluxes (Nesting).
\item AGRIF\_POLY\_DUAVG : Apply a third order polynomial temporal interpolation 
for DU\_avg1 and DU\_avg2 (Nesting).
\item AGRIF\_CORRECT\_LOCAL\_FLUXES  : Apply a local barotropic flux conservation 
enforcement at the child boundaries (Nesting).
\item AGRIF\_RAD2D : Radiate child barotropic boundary conditions (Nesting).
\item AGRIF\_RAD3D : Radiate child 3d momentum boundary conditions (Nesting).
\item AGRIF\_RADT : Radiate child tracer boundary conditions (Nesting).
\item AGRIF\_2WAY : 2-way nesting (Nesting).
\end{itemize}
The default definitions should give satisfaction for most of the applications.
Edit the file AGRIF\_FixedGrids.in. This file contains the child grid positions
(i.e. imin,imax,jmin,jmax) and coefficients of refinement. A first line
gives the number of children grids per parent (if AGRIF\_STORE\_BAROT\_CHILD
is defined, only one child grid can be defined per parent grid). A second
line gives the relative position of each grid and the coefficient of refinement 
for each dimension. 
Edit the input files roms.in.1, roms.in.2 , etc... to define correctly the 
file names and the time steps. To run the model, simply type at the prompt:
roms roms.in.
To visualize the different grid levels, change the value of the "child models" box
in roms\_gui.


