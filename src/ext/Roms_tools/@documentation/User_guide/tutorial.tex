This section presents the essential steps for preparing
and running a regional ROMS simulation. This is 
done following the example of a model of the
Peruvian Upwelling System at low resolution.
 
\subsection{Getting started}

Once the installation has been successful, start a Matlab session
in the directory: $\sim$/Roms\_tools/Run. Launch the start.m
script to set the Matlab path for this session: \\
\\
$>>$ start \\
Add the paths of the different toolboxes... \\
$>>$ \\
\\
You are now ready to start a new configuration.
It is now important to respect the order of the following steps.

\subsection{Building the grid}

Open the file : $\sim$/Roms\_tools/Preprocessing\_tools/make\_grid.m in 
you favorite editor. The part of the file that you should change is :
\\ \\
\%\\
\%  Title\\ 
\%\\
title='Peru Test Model';\\
\%\\
\%  Grid file name\\
\%\\
grdname='roms\_grd.nc';\\
\%\\
\% Slope parameter (r=grad(h)/h) maximum value for topography smoothing\\
\%\\
rtarget=0.2;\\
\%\\
\% Grid dimensions:\\
\%   lonmin : Minimum longitude [degree east]\\
\%   lonmax : Maximum longitude [degree east]\\
\%   latmin : Minimum latitude [degree north]\\
\%   latmax : Maximum latitude [degree north]\\
\%\\
lonmin=-85;\\
lonmax=-75;\\
latmin=-15;\\
latmax=-7;\\
\%\\
\% Grid resolution [degree]\\
\%\\
dl=1/3;\\
\%\\
\% Minimum depth [m]\\
\%\\
hmin=10;\\
\%\\
\%  Topography netcdf file name (ETOPO 2)\\
\%\\
topofile='../Topo/etopo2.nc';\\
\%\\
\%  GSHSS user defined coastline (see m\_map) \\
\%\\
coastfileplot='';\\
\\
The variables that you can change :
\begin{itemize}
\item title='Peru Test Model' : You can give any name 
you want for your configuration
\item grdname='roms\_grd.nc' : Name of the ROMS netcdf grid file  you are 
building. In general, we follow the style : XXX\_grd.nc.
\item rtarget=0.2 : This variable control the maximum value of the $r$-parameter
that measures the slope of the sigma layers \citep{Bec93}:
$$
r=\frac{\nabla h}{2h}=\frac{h_{+1/2}-h_{-1/2}}{h_{+1/2}+h_{-1/2}}  
$$
To prevent horizontal pressure gradients errors, well known in
terrain-following coordinate models \citep{Han91}, realistic topography
requires some smoothing. Empirical results have shown that reliable
results are obtained if $r$ does not exceed 0.2.
\item lonmin=-85 : Western limit of the grid in longitude [-180$^\circ$E, 180$^\circ$E]. 
The grid is rectangular in latitude/longitude.
\item lonmax=-75 : Eastern limit [-180$^\circ$E, 180$^\circ$E]. 
Should be superior to lonmin.
\item latmin=-15 : Southern limit of the grid in latitude [-90$^\circ$N, 90$^\circ$N].
\item latmax=-7 : Northern limit [-90$^\circ$N, 90$^\circ$N].
Should be superior to latmin.
\item dl=1/3 : Grid longitude spacing in degrees. The latitude spacing is deduced to
obtain an isotropic grid ($d\phi=dl\cos(\phi)$).
\item hmin=10 : Minimum depth in meters. The model depth is cut a this level 
to prevent, for example, the occurrence of model grid cells without water.
This does not influence the masking routines.
\item topofile='../Topo/etopo2.nc' : Default topography file. We are using 
here etopo2 \citep{Smi97}. 
\item coastfileplot='' : If you have a m\_map binary coastal file to make 
nice plots, you can add it here. These files can be generated with make\_coast.m.
If there is no file, the program will use a default global coarse coastline file.
This has only an effects on graphics. 
\end{itemize}

Save make\_grid.m and run it in the Matlab session :
\\ \\
$>>$ make\_grid
\\ \\
Be careful to note the values given for LLm and MMm during the process,
they will be necessary for the ROMS file param.h (in this test case,
LLm0 = 29 and MMm0 = 24). Figure (\ref{fig:grid}) presents the
bottom topography obtained with make\_grid.m for the
Peru Upwelling example. Note that at this low
resolution (1/3$^\circ$), the topography has been strongly
smoothed.
Once you are happy with the horizontal grid, you 
can add the surface forcing.

\begin{figure}[h!]
\centerline{\psfig{figure=grid.eps,width=10cm}}
\caption{Result of make\_grid.m for the Peru example}
\label{fig:grid}
\end{figure}

\subsection{Getting the wind and other surface fluxes}

The next step is to create the file containing the different surface
fluxes. Edit the file\\
$\sim$/Roms\_tools/Preprocessing\_tools/make\_forcing.m.
Since we are going to use the default climatology surface forcing \citep{Das94},
the part that you need to change is only: 
\\ \\
\%\\
\%  Title - Grid file name - Forcing file name \\
\% \\
title='Forcing (COADS)'; \\
grdname='roms\_grd.nc'; \\
frcname='roms\_frc.nc'; \\
\% \\
\\
These variables are :
\begin{itemize}
\item title='Forcing (COADS)' : Any title is good.
\item grdname='roms\_grd.nc' : Name of the ROMS netcdf grid file 
that has been previously created. 
\item frcname='roms\_frc.nc' : Name of the ROMS netcdf surface forcing file  you are 
building. In most cases, we follow the style : XXX\_frc.nc.
\end{itemize}

Save make\_forcing.m and run it in the Matlab session :
\\ \\
$>>$ make\_forcing
\\ \\
This program can take a relatively long time to process all the variables.
\begin{figure}[h!]
\centerline{\psfig{figure=sst_wind.eps,width=14cm}}
\caption{Result of make\_forcing.m for the Peru example}
\label{fig:forcing}
\end{figure}
Figure (\ref{fig:forcing}) presents the
sea surface temperature and the wind vectors that has
been interpolated from the climatology \citep{Das94}
at 4 different times of the year.
The sea surface temperature is used for the restoring term
in the heat flux calculation. 

\subsection{Getting the lateral boundary conditions}

The last preprocessing step consist in generating the files containing 
the necessary informations for the lateral open boundaries and the initial
conditions of the model. 
This is done using the file $\sim$/Roms\_tools/Preprocessing\_tools/make\_clim.m.
This script process two files : the climatology file that gives the lateral
boundary conditions and the initial condition file.
The part that can be changed by the user is as follow:
\\ \\
\%\\
\%  Title \\
\%\\
title='Climatology';\\
\%\\
\%  Switches for selecting what to process (1=ON)\\
\%\\
makeclim=1; \%1: process boundary data\\
makeoa=1;   \%1: process oa data\\
makeini=1;  \%1: process initial data\\
\%\\
\%  Grid file name - Climatology file name\\
\%  Initial file name - OA file name\\
\%\\
grdname='roms\_grd.nc';\\
frcname='roms\_frc.nc';\\
clmname='roms\_clm.nc';\\
ininame='roms\_ini.nc';\\
oaname ='roms\_oa.nc';\\
\%\\
\%  Vertical grid parameters\\
\%  !!!!!!!!!!!!!!!!!!!!!!!!!!!!!!!!!!!!!!!!!!!!!!!!\\
\%  !!! WARNING WARNING WARNING WARNING WARNING  !!!\\
\%  !!!   THESE MUST IDENTICAL IN THE .IN FILE   !!!\\
\%  !!! WARNING WARNING WARNING WARNING WARNING  !!!\\
\%  !!!!!!!!!!!!!!!!!!!!!!!!!!!!!!!!!!!!!!!!!!!!!!!!\\
\%\\
theta\_s=7.;\\
theta\_b=0.;\\
hc=5.;\\
N=20; \% number of vertical levels (rho)\\
\%\\
\%  Open boundaries switches\\
\%  !!!!!!!!!!!!!!!!!!!!!!!!!!!!!!!!!!!!!!!!!!!!!!!!\\
\%  !!! WARNING WARNING WARNING WARNING WARNING  !!!\\
\%  !!!   MUST BE CONFORM TO THE CPP SWITCHES    !!!\\
\%  !!! WARNING WARNING WARNING WARNING WARNING  !!!\\
\%  !!!!!!!!!!!!!!!!!!!!!!!!!!!!!!!!!!!!!!!!!!!!!!!!\\
\%\\
obc=[1 0 1 1]; \% open boundaries (1=open , [S E N W])\\
\%\\
\%  Level of reference for geostrophy calculation\\
\%\\
zref=-500; \\
\%\\
\%  Day of initialization\\
\%\\
tini=15;  \\
\%\\
\% Set times and cycles: monthly climatology for all data\\
\%\\
time=[15:30:345];    \% time \\
cycle=360;           \% cycle \\
\%\\
\%  Data climatologies file names:\\
\%\\
\%    temp\_month\_data : monthly temperature climatology\\
\%    temp\_ann\_data   : annual temperature climatology\\
\%    salt\_month\_data : monthly salinity climatology\\
\%    salt\_ann\_data   : annual salinity climatology\\
\%\\
temp\_month\_data='../WOA2001/temp\_month.cdf';\\
temp\_ann\_data='../WOA2001/temp\_ann.cdf';\\
insitu2pot=1;   \% transform in-situ temperature to potential temperature\\
salt\_month\_data='../WOA2001/salt\_month.cdf';\\
salt\_ann\_data='../WOA2001/salt\_ann.cdf';\\
\%\\
\%\\
\\
Variables description :
\begin{itemize}
\item title='Climatology' : Any title is good.
\item makeclim=1 : Switch to define if the climatology
 (boundary conditions) file is generated. should be 1.
\item makeoa=1 : Switch to define if the OA (objective analysis)
 file is generated. This should be 1. The OA files are intermediate files
 where hydrographic data are stored on a ROMS horizontal grid but on
 a z vertical grid. The transformation to S-coordinate is done after.
 These files are not used by ROMS.
\item makeini=1 : Switch to define if the initial file is generated. 
It should be 1.
\item grdname='roms\_grd.nc' : Grid file name previously generated.
\item frcname='roms\_frc.nc' : Surface forcing file name previously 
generated. We need to provide the surface forcing to compute the Ekman 
transport for the velocities of the lateral boundary conditions.
\item clmname='roms\_clm.nc' : Climatology file name.
\item ininame='roms\_ini.nc' : Initial file name.
\item oaname ='roms\_oa.nc' : OA intermediate file name.
\item theta\_s=7. : Vertical S-coordinate surface stretching parameter. 
When building the climatology and initial ROMS files, we have to define
the vertical grid. Be cautious, the vertical grid parameters should be 
identical in this file and in the ROMS input file (i.e. 
$\sim$/Roms\_tools/Run/roms.in).
This is a serious cause of bug.
The effects of theta\_s, theta\_b, hc, and N can be tested 
using the script : \\
$\sim$/Roms\_tools/Preprocessing\_tools/test\_vgrid.m.
\item theta\_b=0. : Vertical S-coordinate bottom stretching parameter.
\item hc=5. : Vertical S-coordinate $H_c$ parameter. It gives approximately the
transition depth between the horizontal surface levels and the bottom terrain following
levels.It should be inferior to hmin.
\item N=20 : Number of vertical level. Be careful, this have to be also 
defined in the file : $\sim$/Roms\_tools/Run/param.h before compiling
the model.
\item obc=[1 0 1 1] : Switches to open (1) or close (0=wall) the lateral
boundaries [South East North West]. This is used when the constraint of mass
enforcement is applied. Be aware, this should be compatible with the open boundary
CPP-switches in the file $\sim$/Roms\_tools/Run/cppdefs.h.
\item zref=-500 : Depth in meters of no motion when computing the geostrophic 
velocities.
\item tini=15 : Day of initialization (15 = January 15).
\item time=[15:30:345] : Time in days in the datasets (Matlab vector).
\item cycle=360 : This variable define that the model will be run 
forced by climatological data on a cyclic year of 360 days.
If not, it should be 0.
\item temp\_month\_data='../WOA2001/temp\_month.cdf' : Name of the monthly
climatology temperature data file.
\item temp\_ann\_data='../WOA2001/temp\_ann.cdf' : Name of the annual
climatology temperature data file. Since the monthly climatology 
contains only the 1500 first meters, the annual climatology is used for the 
deeper levels.
\item insitu2pot=1 : Switch defined if it is in-situ temperature that is provided.
In this case, in-situ temperature is converted into potential temperature.
\item salt\_month\_data='../WOA2001/salt\_month.cdf' : Name of the monthly
climatology salinity data file.
\item salt\_ann\_data='../WOA2001/salt\_ann.cdf' : Name of the annual
climatology salinity data file.
\end{itemize}

Save make\_clim.m and run it in the Matlab session :
\\ \\
$>>$ make\_clim 
\\ \\
This program can also take quite a long time to run.

\begin{figure}[h!]
\centerline{\psfig{figure=section_clim.eps,width=12cm}}
\caption{Result of make\_clim.m for the Peru example}
\label{fig:clim}
\end{figure}

Figure (\ref{fig:clim}) presents 4 different sections
of temperature in the initial condition file for the 
Peru example. The section are in the X-direction (East-West), 
the first section is for the South of the domain and the last one 
is for the North of the domain.

\subsection{Compiling and running the model}

Once all the netcdf data files are ready (i.e. XXX\_grd.nc,
XXX\_frc.nc, XXX\_ini.nc, and XXX\_clm.nc), we can 
prepare ROMS for compilation. All is done in the 
 $\sim$/Roms\_tools/Run/ directory. First edit the file 
 $\sim$/Roms\_tools/Run/param.h.
 The line that needs to be changed is:\\ 
 \\
\#  elif defined PERU

      parameter (LLm0=29,  MMm0=24, N=20)  ! $<--$ Peru Test Case\\
\#  else\\
\\
These are the values of the model grid size: LLm0 points in the X 
direction, MM0m points in the Y direction and N vertical levels.
LLm0 and MMm0 are given by running make\_grid.m, and N is
defined in make\_clim.m.

The second file to change is  $\sim$/Roms\_tools/Run/cppdefs.h.
This file defines the CPP keys that are used by the
the C-preprocessor when compiling ROMS.
Definitions of the CCP keys in cppdefs.h:
\begin{itemize}
\item BASIN     : Must be defined for running the Basin Example.
\item CANYON\_A  : Must be defined for running the Canyon\_A Example.
\item CANYON\_B  : Must be defined for running the Canyon\_B Example.
\item GRAV\_ADJ  : Must be defined for running the Gravitational Adjustment Example.
\item INNERSHELF   : Must be defined for running the Inner Shelf Example.
\item OVERFLOW  : Must be defined for running the Graviational/Overflow Example.
\item SEAMOUNT  : Must be defined for running the Seamount Example.
\item SHELFRONT : Must be defined for running the Shelf Front Example.
\item SOLITON   : Must be defined for running the Equatorial Rossby Wave Example.
\item UPWELLING : Must be defined for running the Upwelling Example.
\item REGIONAL\_MODEL : Must be defined if running realistic regional simulations.
\item PERU :  Configuration Name, this is used in param.h.

\item OPENMP : Activate Open-MP parallelization protocol.
\item MPI : Activate MPI parallelization protocol.
\item AGRIF : Activate the nesting capabilities

\item SOLVE3D : Define if solving 3D primitive equations
\item UV\_COR : Activate Coriolis terms.
\item UV\_ADV : Activate advection terms.
\item SSH\_TIDES : Define for processing sea surface elevation tidal data at the model boundaries.
\item UV\_TIDES :  Define for processing ocean current tidal data at the model boundaries.
\item VAR\_RHO\_2D : Activate nonuniform density in barotropic mode pressure-
 gradient terms.
\item FLAT\_WEIGHTS :  Use a more dissipative averaging for the baroclinic/barotropic coupling.

\item CURVGRID : Activate curvilinear coordinate grid option.
\item SPHERICAL : Activate longitude/latitude grid positioning.
\item MASKING : Activate land masking in the domain.
\item AVERAGES : Define if writing out time-averaged data.
\item SALINITY : Define if using salinity.
\item NONLIN\_EOS : Activate the nonlinear equation of state.
\item SPLIT\_EOS : Activate to split the nonlinear equation of state in a
adiabatic part and a compressible part.

\item ZCLIMATOLOGY : Activate processing of  sea surface height climatology.
\item UCLIMATOLOGY : Activate processing of  momentum climatology.
\item ZNUDGING : Activate open boundary passive/active term + nudging layer for zeta.
\item M2NUDGING : Activate open boundary passive/active term + nudging layer for ubar and vbar.
\item SPONGE : Activate areas of enhanced viscosity/diffusion.

\item QCORRECTION : Activate net heat flux correction.
\item SFLX\_CORR : Activate freshwater flux correction.
\item DIURNAL\_SRFLUX : Activate diurnal modulation of the short wave radiation flux
\item TCLIMATOLOGY : Activate processing of tracer climatology.
\item TNUDGING : Activate open boundary passive/active term + nudging layer for tracers.
\item M3NUDGING : Activate open boundary passive/active term + nudging layer for u and v.
\item ROBUST\_DIAG : Activate strong tracer nudging in the interior for diagnostic 
simulations.

\item ANA\_SSFLUX : Define if using analytical surface salinity flux.
\item ANA\_SST : Define if using analytical SST and dQdSST.
\item ANA\_SSS : Define if using analytical sea surface salinity.
\item ANA\_SRFLUX : Define if using analytical surface shortwave radiation flux.
\item ANA\_STFLUX : Define if using analytical surface temperature flux.
\item ANA\_BSFLUX : Define if using analytical bottom salinity flux.
\item ANA\_BTFLUX : Define if using analytical bottom temperature flux.
\item ANA\_TCLIMA : Activate analytical tracer climatology.
\item ANA\_SMFLUX : Define if using analytical surface momentum stress.
\item ANA\_UCLIMA : Activate analytical momentum climatology.

\item VIS\_GRID : Activate viscosity coefficient scaled by grid size.
\item UV\_VIS2 : Activate Laplacian horizontal mixing.
\item TS\_DIF2 : Activate Laplacian horizontal mixing.
\item DIF\_GRID : Activate diffusion coefficient scaled by grid size.
\item MIX\_GP\_TS : Activate mixing on geopotential (constant Z) surfaces.
\item MIX\_GP\_UV : Activate mixing on geopotential (constant Z) surfaces.
\item CLIMAT\_TS\_MIXH : Activate mixing of T-Tclim instead of tracers.

\item BODYFORCE : Define if applying stresses as bodyforces.
\item BVF\_MIXING : Activate Brunt-Vaisala frequency mixing scheme.
\item PP\_MIXING : Activate Pacanowsky/Philander mixing scheme.
\item LMD\_MIXING : Activate Large/McWilliams/Doney mixing (LMD-KPP closure).
\item LMD\_SKPP : Activate surface boundary layer KPP mixing (LMD-KPP closure). 
\item LMD\_BKPP : Activate bottom boundary layer KPP mixing (LMD-KPP closure). 
\item LMD\_RIMIX : Activate shear instability interior mixing (LMD-KPP closure).
\item LMD\_CONVEC : Activate convection interior mixing (LMD-KPP closure).
\item LMD\_DDMIX : Activate double diffusion interior mixing (LMD-KPP closure). 
\item LMD\_BOTEK :  Activate a bottom Ekman layer parametrisation. 
\item LMD\_NONLOCAL : Activate nonlocal transport (LMD-KPP closure). 

\item OBC\_EAST : Open eastern boundary (should be consistent with make\_clim.m).
\item OBC\_WEST : Open western boundary (should be consistent with make\_clim.m).
\item OBC\_NORTH : Open northern boundary (should be consistent with make\_clim.m).
\item OBC\_SOUTH : Open southern boundary (should be consistent with make\_clim.m).

\item OBC\_VOLCONS : Activate mass conservation enforcement at open boundaries.
\item OBC\_M2ORLANSKI : Activate 2D radiation open boundary conditions for ubar and vbar.
\item OBC\_M2FLATHER :  Activate Flather open boundary conditions for ubar and vbar.
\item OBC\_TORLANSKI :  Activate 2D radiation open boundary conditions for tracers.
\item OBC\_M3ORLANSKI : Activate 2D radiation open boundary conditions for u and v.

\item OBC\_TSPECIFIED : Activate specified open boundary conditions for tracers.
\item OBC\_M2SPECIFIED : Activate specified open boundary conditions for ubar and vbar.
\item OBC\_M3SPECIFIED : Activate specified open boundary conditions for u and v.

\item AGRIF\_STORE\_BAROT\_CHILD : Store ubar and vbar during the parent step for the
chidl boundary conditions (Nesting).
\item AGRIF\_FLUX\_BC : Apply parent/child barotropic boundary conditions has 
fluxes (Nesting).
\item AGRIF\_POLY\_DUAVG : Apply a third order polynomial temporal interpolation 
for DU\_avg1 and DU\_avg2 (Nesting).
\item AGRIF\_CORRECT\_LOCAL\_FLUXES  : Apply a local barotropic flux conservation 
enforcement at the child boundaries (Nesting).
\item AGRIF\_RAD2D : Radiate child barotropic boundary conditions (Nesting).
\item AGRIF\_RAD3D : Radiate child 3d momentum boundary conditions (Nesting).
\item AGRIF\_RADT : Radiate child tracer boundary conditions (Nesting).
\item AGRIF\_2WAY : 2-way nesting (Nesting).

\item BIOLOGY : Activate the biogechemical module.

\item FLOATS : Activate floats.
\item FLOATS\_GLOBAL\_ATTRIBUTES
\item RANDOM\_WALK

\item PASSIVE\_TRACER : Add a passive tracer

\item SEDIMENT : Activate the sediment module.
\item ANA\_SEDIMENT
\item BED\_ARMOR
\item ANA\_SPFLUX
\item ANA\_BPFLUX
\item LINEAR\_CONTINUATION
\item NEUMANN
\item ROUSE

\item BBL : Activate the bottom boundary layer module.
\item ANA\_WWAVE
\item ANA\_BSEDIM
\item Z0\_BL
\item Z0\_RIP
\item Z0\_BIO

\end{itemize}

In the case of standard regional modeling, only open boundary conditions switches
should be changed in cppdefs.h (to be conformed to make\_clim.m). 
Once done, ROMS can be compiled
by running the UNIX script $\sim$/Roms\_tools/Run/jobcomp.
The script jobcomp is able to recognize your system. It has been tested with 
Linux, IBM and Compaq computers. On Linux PCs, the default compiler is the GNU g77, 
but it is possible to uncomment specific lines in jobcomp to use the Intel Fortran
Compiler (IFC). The latter is mandatory when using AGRIF and/or OPEN\_MP.
When changing the compiler you should provide a corresponding netcdf library.
Once compiled you should obtain a new
executable (roms) in the $\sim$/Roms\_tools/Run directory.

Now, edit the input parameter file: $\sim$/Roms\_tools/Run/roms.in.
The vertical grid parameters (THETA\_S,   THETA\_B,   HC)
should be identical to the ones in make\_clim.m. 
Otherwise, the other default values should not be changed.
The meaning of all the input variable is given at the start of each ROMS
simulation.
To run the model, type in directory $\sim$/Roms\_tools/Run/ : roms roms.in.

\subsection{Getting the results}

Once the model has run, or during the simulation, it is possible
to easily visualize the model outputs using a Matlab graphic user interface : 
roms\_gui. Launch roms\_gui in the Matlab session 
(in $\sim$/Roms\_tools/Run/ directory):
\\ \\
$>>$ roms\_gui
\\ \\
\begin{figure}[!ht]
\centerline{\psfig{figure=select.eps,width=7cm}}
\caption{Entrance window of roms\_gui}
\label{fig:open}
\end{figure}

A window pops up, asking for a ROMS history netcdf file (Figure \ref{fig:open}).
You should select roms\_his.nc (history file) or roms\_avg.nc (average file) and
click "open".
 
\begin{figure}[!ht]
\centerline{\psfig{figure=roms_gui2.eps,width=12cm}}
\caption{roms\_gui}
\label{fig:romsgui}
\end{figure}

The main window appears, variables can be selected to obtain an image such as
(Figure \ref{fig:romsgui}). The left box  gives the available variable names.
the right box presents the derived variables. These are variables diagnosed
from ROMS outputs :

\begin{itemize}
\item Ke : Horizontal slice of kinetic energy: $0.5(u^2+v^2)$.

\item Rho : Horizontal slice of density using the non-linear equation of state 
for seawater of \citet{Jac95}. 

\item Vor : Horizontal slice of vorticity: $\frac{\partial v}{\partial x}-
\frac{\partial u}{\partial y}$.

\item Vp : Horizontal slice of the vertical component of Ertel's potential vorticity:
$\frac{\partial \lambda}{\partial z} \left [ f + 
\left (\frac{\partial v}{\partial x}-\frac{\partial u}{\partial y}\right ) \right ]$.
In our case, $\lambda=\rho$.

\item Psi : Horizontal slice of stream function: 
$\nabla^2 \psi=\frac{\partial v}{\partial x}-\frac{\partial u}{\partial y}$.
This routine might be costly since it inverses the Laplacian of the vorticity
(using a successive over relaxation solver).

\item Spd : Horizontal slice of the ocean currents velocity : $\sqrt{u^2+v^2}$.

\item Svd : Horizontal slice of the transport stream function : 
$\nabla^2 S_{vd}=\frac{\partial \bar{v}}{\partial x}-\frac{\partial \bar{u}}{\partial y}$.


\item Oku : Horizontal slice of the Okubo-Weiss parameter : 
$\Lambda^2=
\left ( \frac{\partial u}{\partial x}-\frac{\partial v}{\partial y} \right )^2+
\left ( \frac{\partial v}{\partial x}+\frac{\partial u}{\partial y} \right )^2-
\left ( \frac{\partial v}{\partial x}-\frac{\partial u}{\partial y} \right )^2$.

\end{itemize}

It is possible to add arrows for the horizontal currents by increasing the "Current vectors 
spatial step". It is also possible to obtain vertical sections and 
Hovm\"uller diagrams by clicking on the corresponding targets in roms\_gui.
