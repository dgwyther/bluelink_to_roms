\bibitem[Beckmann and Haidvogel(1993)]{Bec93}
Beckmann, A., Haidvogel, D.B.,
1993.
Numerical simulation of flow around a tall
isolated seamount. Part I: Problem formulation
and model accuracy.
Journal of Physical Oceanography
23,
1736-1753.

\bibitem[Blanke et al.(2002)]{Bla02}
Blanke, B., Roy, C., Penven, P., Speich, S., 
McWilliams, J.C., Nelson, G.,
2002.
Linking wind and upwelling interannual variability in a
regional model of the southern Benguela,
Geophysical Research Letters
29,
2188-2191.

\bibitem[Blayo and Debreu(1999)]{Bla99}
Blayo, E., Debreu, L.,
1999.
Adaptive mesh refinement for finite-difference
ocean models: First experiments.
Journal of Physical Oceanography
29,
1239-1250.

\bibitem[Conkright et al.(2002)]{Con02}
Conkright, M.E., R.A. Locarnini, H.E. Garcia, T.D. O Brien, 
T.P. Boyer, C. Stephens, J.I. Antonov, 2002. 
World Ocean Atlas 2001: Objective Analyses, Data Statistics, 
and Figures, CD-ROM Documentation. 
National Oceanographic Data Center, 
Silver Spring, MD, 
17 pp.

\bibitem[Da Silva et al.(1994)]{Das94}
Da Silva, A.M., Young, C.C., Levitus, S.
1994.
Atlas of surface marine data 1994,
Vol. 1,
algorithms and procedures,
NOAA Atlas NESDIS 6,
U. S. Department of Commerce,
NOAA,
NESDIS,
USA,
74 pp.

\bibitem[Debreu(2000)]{Deb00}
Debreu, L.,
2000.
Raffinement adaptatif de maillage et m\'ethodes de zoom -
application aux mod\`eles d'oc\'ean,
2000,
Ph.D. thesis,
Universit\'e Joseph Fourier,
Grenoble.

\bibitem[Debreu and Blayo(2003)]{Deb03a}
Debreu, L., Blayo, E.,
2003.
AGRIF: Adaptive Grid Refinement In Fortran.
{\it submitted to ACM Transactions on Mathematical Software - TOMS}.

\bibitem[Debreu and Vouland(2003)]{Deb03b}
Debreu, L., Vouland, C.,
2003.
AGRIF: Adaptive Grid Refinement In Fortran.
[Available online http://www-lmc.imag.fr/IDOPT/AGRIF/index.html].

\bibitem[Di Lorenzo et al.(2003)]{Dil03}
Di Lorenzo, E., Miller, A.J., Neilson, D.J., 
Cornuelle, B.D., Moisan, J.R.,
2003.
Modeling observed California Current mesoscale eddies and 
the ecosystem response .
International Journal of Remote Sensing,
in press.

\bibitem[Egbert and Erofeeva(2002)]{Egb02}
Egbert, G., Erofeeva, S., 2002.
Efficient inverse modeling of barotropic ocean tides, 
Journal of Atmospheric and Oceanic Technology
19,
183-204.

\bibitem[Flather(1976)]{Fla76}
Flather, R.A., 1976.
A tidal model of the northwest European continental shelf.
M\'emoires de la Soci\'et\'e Royale des Sciences de Li\`ege, 
10, 
141-164.

\bibitem[Haidvogel et al.(2000)]{Hai00}
Haidvogel, D.B., Arango, H.G., Hedstrom, K. , Beckmann, A.,
Malanotte-Rizzoli, P., Shchepetkin, A.F.,
2000.
Model Evaluation Experiments in the North Atlantic Basin:
Simulations in Nonlinear Terrain-Following Coordinates.
Dynamics of Atmospheres and Oceans
32 ,
239-281.

\bibitem[Haney(1991)]{Han91}
Haney, R.L.,
1991.
On the pressure force over steep
topography in sigma coordinate ocean models.
Journal of Physical Oceanography
21,
610-619.

\bibitem[Jackett and McDougall(1995)]{Jac95}
Jackett, D.R., McDougall, T.J.,
1995.
Minimal Adjustment of Hydrostatic Profiles to 
Achieve Static Stability.
Journal of Atmospheric and Oceanic Technology
12,
381-389.

\bibitem[Large(1994)]{Lar94}
Large, W.G., McWilliams, J.C., Doney, S.C.,
1994.
Oceanic vertical mixing: a review and a model
with a nonlocal boundary layer parameterization.
Reviews in Geophysics
32,
363-403.

\bibitem[MacCready et al.(2002)]{Mac02}
MacCready, P. M., R. D. Hetland, W. R. Geyer, 
Long-Term Isohaline Salt Balance in an Estuary. 
Continental Shelf Research, 22, 1591-1601.

\bibitem[Marchesiello et al.(2001)]{Mar01}
Marchesiello, P., McWilliams, J.C., Shchepetkin, A.,
2001.
Open boundary condition for long-term integration of regional oceanic
models.
Ocean Modelling
3,
1-21.

\bibitem[Marchesiello et al.(2003)]{Mar03}
Marchesiello, P., McWilliams, J.C., Shchepetkin, A.,
2003.
Equilibrium structure and dynamics of the California Current System.
Journal of Physical Oceanography
33,
753-783.

\bibitem[Penven et al.(2001)]{Pen01}
Penven, P., Roy C., Lutjeharms, J.R.E., 
Colin de verdi\`ere, A., Johnson, A., Shillington, F.,
Fr\'eon, P., Brundrit, G.,
2001.
A regional hydrodynamic model of the Southern Benguela.
South African Journal of Science
97,
472-476.

\bibitem[Penven et al.(2004)]{Pen04}
Penven, P., Debreu, L., Marchesiello,  P., McWilliams, J.C.,
2004.
Application of the ROMS embedding procedure for the Central 
California Upwelling System.
Ocean Modelling, 
In revision.

\bibitem[Shchepetkin and McWilliams(1998)]{Shc98}
Shchepetkin, A.F., McWilliams, J.C.,
1998.
Quasi-monotone advection schemes based on explicit locally
adaptive dissipation.
Monthly Weather Review
126,
1541-1580.

\bibitem[Shchepetkin and McWilliams(2003a)]{Shc03a}
Shchepetkin, A.F., McWilliams, J.C.,
2003.
A method for computing horizontal pressure-gradient force
in an ocean model with a non-aligned vertical coordinate.
Journal of Geophysical Research
108.

\bibitem[Shchepetkin and McWilliams(2004)]{Shc03b}
Shchepetkin, A.F., McWilliams, J.C.,
2004.
Regional Ocean Model System: a split-explicit ocean model with a
free-surface and topography-following vertical coordinate.
Submitted to Journal of Computational Physics.

\bibitem[Smith and Sandwell(1997)]{Smi97}
Smith, W.H.F., Sandwell, D.T.,
1997.
Global seafloor topography from satellite altimetry and ship depth soundings.
Science
277,
1957-1962.
